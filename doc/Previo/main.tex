\documentclass[a4paper,12pt]{article}

% -----------------------
% Paquetes
% -----------------------
\usepackage[utf8]{inputenc}
\usepackage[spanish]{babel}
\usepackage{amsmath}
\usepackage{listings}
\usepackage{xcolor}
\usepackage{geometry}
\usepackage{graphicx}
\geometry{margin=2.5cm}

% Configuración de código
\lstset{
    language=Python,
    backgroundcolor=\color{gray!10},
    basicstyle=\ttfamily\small,
    keywordstyle=\color{blue}\bfseries,
    commentstyle=\color{green!50!black},
    stringstyle=\color{red!70!black},
    showstringspaces=false,
    frame=single
}

% Definición de entorno opcional (en color púrpura)
\newenvironment{opcional}{
  \par\color{purple}\begin{quote}\textbf{Sección Opcional / Ampliación:}\\
}{
  \end{quote}\par
}

% -----------------------
% Datos para portada
% -----------------------
\title{\Huge Informe sobre Programación Orientada a Objetos en Python \\[0.2cm]
\large Herencia, Clases Abstractas, Polimorfismo, Interfaces y MRO}
\author{Nicolás Castillo Azócar \\[0.2cm]
\textbf{Profesor:} Guido Mellado \\[0.2cm]
\textbf{Asignatura:} Programación II}
\date{\today}

\begin{document}
% Portada
\maketitle
\thispagestyle{empty}
\newpage

% Índice
\tableofcontents
\newpage

% -----------------------
% Introducción
% -----------------------
\section{Introducción}
La Programación Orientada a Objetos (POO) es un paradigma de programación que busca modelar el software utilizando conceptos inspirados en el mundo real, como clases y objetos.  
Los pilares de la POO incluyen la abstracción, la herencia, el polimorfismo y el encapsulamiento.  

En este informe se abordarán los siguientes conceptos clave:
\begin{itemize}
    \item \textbf{Herencia}: reutilización de código a través de clases derivadas.
    \item \textbf{Clases Abstractas}: definición de plantillas obligatorias para subclases.
    \item \textbf{Polimorfismo}: capacidad de usar métodos comunes en diferentes tipos de objetos.
    \item \textbf{Interfaces}: contratos que garantizan consistencia entre distintas clases.
    \item \textbf{Method Resolution Order (MRO)}: mecanismo de resolución de métodos en herencia múltiple.
\end{itemize}

Adicionalmente, se incluirán secciones opcionales sobre \texttt{super()}, \textit{duck typing} y \textit{dunder methods}, útiles para enriquecer el dominio de POO en Python.

% -----------------------
% Herencia
% -----------------------
\section{Herencia}
La \textbf{herencia} permite que una clase hija (o subclase) adquiera atributos y métodos de una clase padre (o superclase). Esto fomenta la reutilización de código y la jerarquía lógica de los objetos.

Ejemplo básico:

\begin{lstlisting}
class Persona:
    def presentacion(self):
        print("Soy una persona")

class Estudiante(Persona):
    def presentacion(self):
        # Uso de super() para extender el comportamiento
        super().presentacion()
        print("...y también soy estudiante")

e = Estudiante()
e.presentacion()
# Salida:
# Soy una persona
# ...y también soy estudiante
\end{lstlisting}

La relación entre una clase y su subclase suele representarse con el principio \textit{“es-un”}.  
Por ejemplo: un estudiante \textbf{es una} persona, un círculo \textbf{es una} figura geométrica.

% -----------------------
% Clases abstractas
% -----------------------
\section{Clases Abstractas}
Las \textbf{clases abstractas} son aquellas que no pueden ser instanciadas directamente.  
Sirven como plantillas que obligan a las subclases a implementar ciertos métodos. En Python se definen con el módulo \texttt{abc}.

\begin{lstlisting}
from abc import ABC, abstractmethod

class Figura(ABC):
    @abstractmethod
    def area(self):
        pass

class Circulo(Figura):
    def __init__(self, radio):
        self.radio = radio
    def area(self):
        return 3.14 * self.radio ** 2
\end{lstlisting}

Esto asegura que toda clase derivada de \texttt{Figura} debe implementar el método \texttt{area()}.

% -----------------------
% Polimorfismo
% -----------------------
\section{Polimorfismo}
El \textbf{polimorfismo} consiste en la capacidad de utilizar un mismo método en diferentes clases, adaptando su comportamiento.

\subsection*{Ejemplo de polimorfismo con clases}
\begin{lstlisting}
class Perro:
    def sonido(self): return "Guau!"

class Gato:
    def sonido(self): return "Miau!"

animales = [Perro(), Gato()]
for animal in animales:
    print(animal.sonido())
# Salida: Guau! / Miau!
\end{lstlisting}

\subsection*{Polimorfismo en funciones internas}
En Python, funciones como \texttt{len()} son polimórficas: pueden operar con strings, listas, tuplas o diccionarios.

% -----------------------
% Interfaces
% -----------------------
\section{Interfaces}
En lenguajes como Java, las interfaces son tipos que definen un conjunto de métodos sin implementación.  
En Python, se simulan con clases abstractas.

\begin{lstlisting}
class Volador(ABC):
    @abstractmethod
    def volar(self): pass

class Pajaro(Volador):
    def volar(self): print("El pájaro vuela")
\end{lstlisting}

También existen alternativas modernas como los \textbf{protocolos}, que permiten declarar interfaces de forma implícita.

% -----------------------
% MRO
% -----------------------
\section{Method Resolution Order (MRO)}
Cuando existe herencia múltiple, Python utiliza el algoritmo \textbf{C3 linearization} para definir el orden de búsqueda de métodos.

\begin{lstlisting}
class A: pass
class B(A): pass
class C(A): pass
class D(B, C): pass

print(D.mro())
# [D, B, C, A, object]
\end{lstlisting}

Este orden garantiza coherencia y evita ambigüedades.  
En general, Python busca de izquierda a derecha en la declaración de herencia.

% -----------------------
% Ejemplo matemático
% -----------------------
\section{Ejemplo Matemático en POO}
Un ejemplo clásico es calcular el área de un círculo. La fórmula es:

\[
A = \pi r^2
\]

En Python:
\begin{lstlisting}
import math
print(math.pi * 3**2)  # 28.27
\end{lstlisting}

% -----------------------
% Esquema
% -----------------------
\section{Esquema Conceptual}
El siguiente esquema muestra la relación de conceptos:

\[
\text{Clases} \rightarrow \text{Herencia} \rightarrow \text{Polimorfismo}
\]
\[
\text{Clases Abstractas} \Rightarrow \text{Interfaces}
\]

% -----------------------
% Conclusión
% -----------------------
\section{Conclusión}
Los conceptos de POO permiten construir programas más claros, reutilizables y fáciles de mantener.  
Python ofrece mecanismos simples pero potentes para aplicar herencia, abstracción, polimorfismo e interfaces.  
El conocimiento del MRO resulta clave al trabajar con herencia múltiple.

% -----------------------
% OPCIONALES
% -----------------------
\section*{Temas Opcionales (Ampliación)}

\begin{opcional}
\subsection*{La función \texttt{super()}}
Permite llamar métodos de la superclase desde la subclase. Se usa principalmente en constructores y para extender métodos ya existentes.
\end{opcional}

\begin{opcional}
\subsection*{Duck Typing}
El \textit{duck typing} se basa en la frase: ``Si camina como un pato y suena como un pato, probablemente sea un pato''.  
En Python importa el comportamiento, no el tipo.
\begin{lstlisting}
def hazlo_volar(obj):
    obj.volar()

class Avion:
    def volar(self): print("El avión vuela")

class SuperHeroe:
    def volar(self): print("El superhéroe vuela")

hazlo_volar(Avion())
hazlo_volar(SuperHeroe())
\end{lstlisting}
\end{opcional}

\begin{opcional}
\subsection*{Dunder Methods}
Los \textit{dunder methods} (doble subrayado, como \_\_init\_\_) permiten redefinir comportamientos internos de Python.
\begin{lstlisting}
class Vector:
    def __init__(self, x, y):
        self.x, self.y = x, y
    def __add__(self, otro):
        return Vector(self.x + otro.x, self.y + otro.y)

v1 = Vector(1,2)
v2 = Vector(3,4)
v3 = v1 + v2
\end{lstlisting}
Aquí redefinimos el operador \texttt{+} gracias al método especial \texttt{\_\_add\_\_}.
\end{opcional}

\end{document}
